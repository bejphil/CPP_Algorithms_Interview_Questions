\documentclass[letterpaper,11pt]{article}
\usepackage[margin=1in]{geometry}
\usepackage{amsmath, amssymb, setspace, esint}
\usepackage{algorithm}
\usepackage{graphicx}
\usepackage{placeins}
\usepackage{algpseudocode}
\usepackage{tikz}

\usepackage{fontspec}
\usepackage{minted}

\allowdisplaybreaks[1]

\makeatletter
\newcommand\blx@unitmark{23sp}
\makeatother

\newcommand{\LOne}[1]{ \lvert \lvert #1 \rvert \rvert_1 }
\newcommand{\LTwoSqr}[1]{ \lvert \lvert #1 \rvert \rvert_2^2 }
\newcommand{\LInf}[1]{ \lvert \lvert #1 \rvert \rvert_{\infty} }

\newcommand\tikzmark[1]{%
  \tikz[remember picture,overlay]\node[inner sep=2pt] (#1) {};}
\newcommand\DrawBox[3][]{%
  \tikz[remember picture,overlay]\draw[#1] ([xshift=-3.5em,yshift=7pt]#2.north west) rectangle (#3.south east);}

\algnewcommand\algorithmicinput{\textbf{Input:}}
\algnewcommand\Input{\item[\algorithmicinput]}

\algnewcommand\algorithmicoutput{\textbf{Output:}}
\algnewcommand\Output{\item[\algorithmicoutput]}

\algdef{SE}[DOWHILE]{Do}{doWhile}{\algorithmicdo}[1]{\algorithmicwhile\ #1}%

\begin{document}

\section*{Algorithm to compute  $\pi$}

\subsection*{Question 1:}

Construct an algorithm in C++ to approximate $\pi$. Assume that you are working on a project
 with a small development team and your algorithm will be used
 as a part of a larger code base.

\begin{itemize}
  \item Assume that the required precision of the calculation is not known until run-time.
  \item You do not need to determine the error $\lvert x - \pi \rvert$ associated with
   your calculation.
  \item Your algorithm should be able to work with single or double-precision floating point numbers
  \item Assume you have access to a compiler that supports any version of C and/or C++ that you would like
  numbers.
\end{itemize}

You can use any method, libraries etc. that you would
like with the exception of functions that approximate transcendental numbers ( e.g.
 std::asin, M\_PI or similar ).

It might be useful to know that $\pi = 4 \sum_{k=1}^{\infty} \frac{ (-1)^{k+1} }{ 2k - 1 }$
( known as the Gregory Series ). The first few terms in the series are $
4( 1 - \frac{1}{3} + \frac{1}{5} - \frac{1}{7} + \dots )$.

This is not the only way to compute $\pi$ and there is no special reason why this
 method should be preferred over any other method.

You do need to write a Make file, compiler/linker commands or
explicitly write out \#include statements, but you are responsible for appropriately
  using namespaces.

\subsection*{Possible Answer 1:}

\subsubsection*{Solution using Gregory Series:}
Start with the identity $\pi \approx 4 \sum_{k=0}^{N} \frac{ (-1)^{k+1} }{ 2k - 1 }$

\begin{minted}[mathescape,
               linenos,
               numbersep=5pt,
               gobble=2,
               frame=lines,
               framesep=2mm]{cpp}

    \* run_time_pi.h *\

    namespace wbp {

    template< typename T >
    T partial_sum( T (*f)(uint index), uint sum_start, uint sum_end ) {

       T sum_total = static_cast<T>( 0 );

       for( uint i = sum_start; i <= sum_end ; i++ ) {
           sum_total += f(i);
       }

       return sum_total;

    }

    template < typename T >
    T alt_sign( uint k ) {
       return ( ( k + 1 )%2 == 0 )?( static_cast<T>( 1 ) ):( static_cast<T>( -1 ) );
    }

    template < typename T >
    T gregory_term( uint k ) {

       return alt_sign<T>( k )*static_cast<T>(1) / static_cast<T>( 2*k - 1 );
    }

    template< typename T >
    T gregory_pi( uint num_iter ) {

      static_assert( std::is_floating_point<T>::value,
       "Template parameter must be a floating-point value.");
      return static_cast<T>( 4 )*partial_sum( &gregory_term<T>, 1, num_iter );

    }

    }
\end{minted}

\subsubsection*{Solution using Power Series:}
Start with the identity $\pi \approx \sqrt{ 6 \sum_{k=1}^{N} \frac{1}{k^2} }$

\begin{minted}[mathescape,
               linenos,
               numbersep=5pt,
               gobble=2,
               frame=lines,
               framesep=2mm]{cpp}

    \* run_time_pi.h *\

    namespace wbp {

    template< typename T >
    T partial_sum( T (*f)(uint index), uint sum_start, uint sum_end ) {

       T sum_total = static_cast<T>( 0 );

       for( uint i = sum_start; i <= sum_end ; i++ ) {
           sum_total += f(i);
       }

       return sum_total;

    }

    template < typename T >
    T pow_term( uint k ) {

        return static_cast<T>( 1 )/static_cast<T>( k * k );

    }

    template< typename T >
    T power_pi( uint num_iter ) {

        static_assert( std::is_floating_point<T>::value,
        "Template parameter must be a floating-point value.");
        return std::sqrt( static_cast<T>( 6 )*partial_sum( &pow_term<T>, 1, num_iter ) );

    }

    }
\end{minted}

\subsection*{Question 2:}

Suppose your team notices that your algorithm is a performance bottle-neck,
how would you speed it up?

\subsection*{Possible Answer 2:}

For any of these series methods the terms in the series are independent, as
 they only depend on index. The series terms can be computed in a canonical ``for''
  loop and easily parallelized using OpenMP for similar.

For a method like the Monte Carlo method each point can be assigned a position
independently of one another.

\subsection*{Question 3:}

Suppose someone else in your team has written an algorithm to compute $\pi$
 using the Gregory Series $ \left( \pi = 4 \sum_{k=0}^{\infty} \frac{ (-1)^{k+1} }{ 2k - 1 } \right)$.

Their algorithm, in pseudo-code, takes the following form:

 \FloatBarrier
 \begin{algorithm}[!htbp]
   \caption{ Estimate $\pi$ using Gregory Sum formula, using pairs of terms.}
   \begin{algorithmic}[1]
    \Input{$n \in \mathbb{N} $}\Comment{Specify the number of terms in the series -- must be an even number}
    \State $x \gets 0$\Comment{ Initialize the total sum to zero. }
     \For{ $k \in [ 1, 3, 5, \dots, n/2 - 1 ]$ }
      \State $x \gets x + \left( \frac{1}{2k-1} - \frac{1}{2( k + 1 ) -1} \right)$
     \EndFor
   \end{algorithmic}
   \Return $4 x$
 \end{algorithm}
 \FloatBarrier

The first few terms using this algorithm would read $
4( ( 1 - \frac{1}{3} ) + ( \frac{1}{5} - \frac{1}{7} ) + \dots )$.

You notice that after a large number of iterations this algorithm exhibits some strange
 behavior -- it does not seem to converge to $\pi$. You decide to investigate further
 and notice the error term $\lvert x - \pi \rvert$ does not go to zero. What
  might be causing this problem?

\subsection*{Possible Answer 3:}

As $k$ grows large we will find that $\frac{1}{2k-1} \approx \frac{1}{2( k + 1 ) -1}
\leftrightarrow \frac{1}{2k-1} - \frac{1}{2( k + 1 ) -1} \approx 0$.
We also expect that these terms will have similar numbers of significant digits --
 this is a situation that will almost certainly result in loss of significance /
 catastrophic cancellation. The result of this ( might ) be the error term
 converging to some non-zero constant, or diverging.

\subsection*{Question 4:}

Suppose your team decides that the number of iterations/recursions necessary
to guarantee required precision can be determined
 at compile time. Further it has been determined that
 double-precision floating-point numbers are the only type that needs to be
 supported.

 How would you re-write your algorithm to be
 computed at compile time instead of run time?

\subsection*{Possible Answer 4:}

Possible answer if starting with the Gregory Sum is stated below -- answer for
other series methods are similar.

\begin{minted}[mathescape,
               linenos,
               numbersep=5pt,
               gobble=2,
               frame=lines,
               framesep=2mm]{cpp}

    \* meta_pi.h *\
    namespace wbp {

    // General case
    template<unsigned int N, typename T = double>
    struct GregorySum {
        static constexpr T sum =
         ( (N % 2) ? 1.0 : -1.0 ) / ( 2*N - 1 ) +  GregorySum<N - 1, T>::sum;
    };

    // Specialized stop case
    template<typename T>
    struct GregorySum<1, T> {
        static constexpr T sum = 1;
    };

    template< unsigned int N >
    double compute_meta_pi() {
        return 4.0*GregorySum< N >::sum;
    }

    }
\end{minted}

Usage: \verb|wbp::compute_meta_pi<101>()|

\end{document}
